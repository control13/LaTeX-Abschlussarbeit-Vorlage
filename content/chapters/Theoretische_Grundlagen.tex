
\chapter{Theoretische Grundlagen} % (fold)
\label{cha:theoretische_grundlagen}

Es wird eine Methode aus \cite{knuth1997art} verwendet. Dafür existiert eine \gls{acro:API}. Die Gleichung \ref{eq:euleridentity} beschreibt das Model. Eine weitere Artz zu Zitieren ist \textcite{weise2014benchmarking}.

\begin{equation}\label{eq:euleridentity}
    e^{i\pi} = -1
\end{equation}
