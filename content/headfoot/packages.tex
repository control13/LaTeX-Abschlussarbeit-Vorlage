% Anpassung an Landessprache
\usepackage[ngerman]{babel}
 
% Verwenden von Sonderzeichen und Silbentrennung
\usepackage[utf8]{inputenc}
\usepackage[T1]{fontenc}
% \DeclareUnicodeCharacter{0008}{}
% \DeclareUnicodeCharacter{00A0}{~}
\usepackage{textcomp} 					% Euro-Zeichen und andere
\usepackage[babel,german=quotes]{csquotes}		% Anf\"uhrungszeichen
\RequirePackage[ngerman=ngerman-x-latest]{hyphsubst} 	% erweiterte Silbentrennung

\usepackage{multicol}                   %

% Befehle aus AMSTeX f\"ur mathematische Symbole z.B. \boldsymbol \mathbb
\usepackage{amsmath,amsfonts,amssymb,amsthm}

% Zeilenabst\"ande und Seitenr\"ander 
\usepackage{setspace}
\usepackage[left=3cm,right=3cm]{geometry}

% Einbinden von JPG-Grafiken
\usepackage{graphicx}

% zum Umflie\ss{}en von Bildern
% Verwendung unter http://de.wikibooks.org/wiki/LaTeX-Kompendium:_Baukastensystem#textumflossene_Bilder
\usepackage{floatflt}

% Eigene Flie\ss{}umgebungen erzeugen
\usepackage{float}


% Verwendung von vordefinierten Farbnamen zur Colorierung
% Palette und Verwendung unter http://kitt.cl.uzh.ch/kitt/CLinZ.CH/src/Kurse/archiv/LaTeX-Kurs-Farben.pdf
\usepackage{xcolor} 

% Tabellen
\usepackage{array}
\usepackage{longtable}
\usepackage{multirow}
\usepackage[normalem]{ulem}                   %

% einfache Grafiken im Code
% Einf\"uhrung unter http://www.math.uni-rostock.de/~dittmer/bsp/pstricks-bsp.pdf
\usepackage{pstricks}

\usepackage{pst-plot}					% Darstellung von Graphen
\usepackage{pstricks-add}				%
\usepackage{pst-tree}					% Darstellung von B\"aumen

%--- Definitionen mit Schattierung
\usepackage{shadethm}                   %

% Quellcodeansichten
\usepackage{verbatim}
\usepackage{moreverb} 					% f\"ur erweiterte Optionen der verbatim Umgebung
% Befehle und Beispiele unter http://www.ctex.org/documents/packages/verbatim/moreverb.pdf
\usepackage{listings} 					% f\"ur angepasste Quellcodeansichten siehe
% Kurzeinf\"uhrung unter http://blog.robert-kummer.de/2006/04/latex-quellcode-listing.html

% Glossar und Abbildungsverzeichnis
\usepackage[
nonumberlist, 						%keine Seitenzahlen anzeigen
acronym,      						%ein Abk\"urzungsverzeichnis erstellen
toc          						%Eintr\"age im Inhaltsverzeichnis
]      							%im Inhaltsverzeichnis auf section-Ebene erscheinen
{glossaries}



% Stichwortverzeichnis
\usepackage{makeidx}					% Erzeugung eines Stichwortverzeichnisses

% verlinktes und Farblich angepasstes Inhaltsverzeichnis
\usepackage[colorlinks=true,
linkcolor=InterneLinkfarbe,
urlcolor=ExterneLinkfarbe]{hyperref}
\usepackage[all]{hypcap}

% URL verlinken, lange URLs umbrechen
\usepackage{url}

% sorgt daf\"ur, dass Leerzeichen hinter parameterlosen Makros nicht als Makroendezeichen interpretiert werden
\usepackage{xspace}

% Beschriftungen f\"ur Abbildungen und Tabellen
\usepackage{caption}
\usepackage{subcaption}

% Entwicklerwarnmeldungen entfernen
\usepackage{scrhack}

\usepackage{algorithm}
\usepackage{algpseudocode}

\usepackage{calc}


\usepackage{hyperref}
%\usepackage{hdvips}					% soll das Problem verhindern, dass Akronyme nicht umgebrochen werden

\usepackage{fancyvrb}
\usepackage{fancyhdr}

\usepackage[backend=biber,natbib=true,style=ieee,dashed=false,date=long,url=false,doi=false,isbn=false]{biblatex}
% \bibliography{content/headfoot/library,content/headfoot/mylibrary}
\addbibresource{content/headfoot/library.bib}
\addbibresource{content/headfoot/mylibrary.bib}
% \bibliographystyle{plain}

\usepackage{blindtext}

% for using R Sweave
\usepackage{Sweave}
